\section{Introduction}
The topic of this report is the Kosterlitz-Thouless phase transition and the XY model. The XY model is a two-dimensional spin model that exhibits a phase transition known as the Kosterlitz-Thouless transition. This transition is characterized by the unbinding of vortex-antivortex pairs, leading to a change in the behavior of the system. In this report, we will explore the theoretical basis of the XY model and the Kosterlitz-Thouless transition, as well as the methods used to study these phenomena and the results obtained from our simulations. Generally two algorithms are used to simulate the XY model: the Metropolis algorithm and the Wolff cluster algorithm. We will compare the results obtained from both algorithms and discuss their advantages and disadvantages in the context of simulating the XY model. In our simulations we will especially focus on the temperature dependence of magnetization and susceptibility, as well as determining the critical temperature of the Kosterlitz-Thouless transition. Finally, we will summarize our findings and discuss their implications for understanding phase transitions in two-dimensional systems. Additionally the methode of finite size scaling will be used to analyze the results and extract critical exponents associated with the Kosterlitz-Thouless transition. For higher dimensions the complexity of the XY model increases and the nature of the phase transition changes, which is why we will only shortly discuss it for higher dimensions.