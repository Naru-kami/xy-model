\section{Results}
We will now investigate the response of the system to changes in various parameters.
Throughout the investigation, physical constants such as $K_B$ and $J$ will be set to 1 for simplicity.

\subsection{Temperature dependence}
\noindent To quantify the temperature dependency of observables, the measurements have been performed with some considerations in mind.
For the system to equilibrate, the first $N_{Burn}=512$ steps have been discarded before recording any data.
The data recording itself involved a total of $N_{Steps}=512$ steps after reaching thermal equilibrium.
Upon adjusting the temperature, the system was again given time to equilibrate again.
The temperature has been set at equidistant points in the interval $T \in (0, 2]$, with $\Delta T = 0.02$.
To perform sweep over the temperature grid, the system will start at the hot state, and gradually decrease in temperatre.
The system will only be initialized to a random spin configuration when starting a new temperature sweep.
It will not be reinitialize after each temperature step $\Delta T$.
This choice has been made deriberately, due to the large correlation time of specifically the Metropolis algorithm at lower temperatures, which would require larger $N_{Burn}$ to reach thermal equilibrium, especially when starting from a hot state.
It would mean that $T_i$ and $T_{i+1}$ are slightly correlated, however by choosing $N_{Burn}$ appropriately, this correlation can be kept small, as seen in later sections.
The entirety of the measurement comprises a total 20 temperature sweeps for all grid sizes in $L={8, 16, 32, 64, 128, 256}$.
\begin{figure}
    \includegraphics[width=0.466\textwidth]{../figs/Metropolis_E.pdf}
    \includegraphics[width=0.466\textwidth]{../figs/Metropolis_C.pdf}
    \includegraphics[width=0.466\textwidth]{../figs/Metropolis_M.pdf}
    \includegraphics[width=0.466\textwidth]{../figs/Metropolis_X.pdf}
    \caption{\label{fig:Metropolis_Observables}Temperature dependency of the observables using the Metropolis algorithm}
\end{figure}
\bigskip\\
Fig.~\ref{fig:Metropolis_Observables} displays the temperature dependency of all four observables performed using the Metropolis algorithm.
When all spins are aligned at $T=0$, $\langle E \rangle$ and $\langle M \rangle$ reach their expected value of $-2$ and $1$ respectively.
Signs of a phase transition can be seen around temperatures where $C_v$ peaks and $chi$ diverges, and $\langle E \rangle$ and $\langle M \rangle$ show (strong) slope changes.
For higher lattice sizes, these features become even more pronounced, which is due to the finite size effects.
At higher temperatures, the spin alignment is almost completely broken, indicated by $\langle E \rangle$ and $\langle M \rangle$ approching 0.
\begin{figure} 
    \includegraphics[width=0.466\textwidth]{../figs/Wolff_E.pdf}
    \includegraphics[width=0.466\textwidth]{../figs/Wolff_C.pdf}
    \includegraphics[width=0.466\textwidth]{../figs/Wolff_M.pdf}
    \includegraphics[width=0.466\textwidth]{../figs/Wolff_X.pdf}
    \caption{\label{fig:Wolff_Observables}Temperature dependency of the observables using the Wolff algorithm}
\end{figure}
\bigskip\\
Fig.~\ref{fig:Wolff_Observables} displays the temperature dependency for the same observables performed using the Wolff algorithm.
The same behavior can be can be seen for this case as well.
The main difference lies in the reduced variance of the measurements, which is a consequence of the reduced correlation time of the Wolff algorithm.
Furthermore, the peaks for the susceptibility are more pronounced and shifted more towards the theoretical values for $T_c$.
The single spin updates of the Metropolis algorithm are heavily affected by critical slowing down, resulting in a larger correlation time and thus higher variance of the measurements.

\subsection{Autocorrelation time}
We have already seen the effects of autocorrelation times on the measurements in the previous subsection, and now we want to quantify them.
The autocorrelation function is given by eq.~(\ref{eq:Autocorrelation_Function}), which is used to measure the time steps needed for two spin configurations to become statistically independent, accoring to eq.~(\ref{eq:Autocorrelation_Time}).
In order to accurately compare the two algorithm, we defined a time step for an algorithm as the attempt to flip all the spins on the lattice.
At a given temperature, the system will be initialized to a cold state for quicker equilibration in the burn-in phase with $N_{Burn}=500$.
After reaching thermal equilibrium, a total of $N_{Meas}=5000$ samples are taken to make sure that the autocorrelation time $\tau$ is well within that range, and for a more accurate estimation of the mean of the observable.
The absolute value of the magnetization $\left| M \right|$ is chosen as the observable.
Then, the autocorrelation function is calculated for all time steps until $\Gamma_M{t} < e^{-3}$, after which there was mostly just noise.
Such a measurement is then repeated $N_{Rep}=20$ times.
This will be repeated for all temperatures in $T \in [0.5, 1.5]$ with 40 temperature points inside, and all lattice sizes in $L={8, 16, 32, 64, 128}$.
\begin{figure}[b]
    \includegraphics[width=0.466\textwidth]{../figs/Autocorrelation_Metropolis.pdf}
    \caption{\label{fig:Autocorrelation_Metropolis}Autocorrelation values as a function of time steps for the Metropolis algorithm at $L=128$.}
    \includegraphics[width=0.466\textwidth]{../figs/Autocorrelation_time_Metropolis.pdf}
    \caption{\label{fig:Autocorrelation_Time_Metropolis}Autocorrelation times as a function of temperature for the Metropolis algorithm for different lattice sizes.}
\end{figure}
\bigskip\\
Fig~(\ref{fig:Autocorrelation_Metropolis})
